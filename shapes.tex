\section{\href{https://github.com/d3/d3-shape}{Shapes}}

\textit{Each of the following contain one eponymous top-level function that produces a \say{generator} which is a late-binding function that creates the indicated shape when called. Input to each generator is a data array. Output shapes are coded as \say{path calls,} which are either svg or html canvas commands (see \href{https://observablehq.com/@d3/d3-line}{here}) depending on the passed }\texttt{<shape>.context()}\textit{.}


%%%%%%%%%%%%%%%%%%%%%%%%%%%%%%%%%%%%%%%%%%%%%%%%%%%
\subsec{\href{https://github.com/d3/d3-path}{Paths}}{<path>\textsubscript{G}}

\textit{Generates: a serialized set of \href{https://observablehq.com/@d3/d3-path}{accumulated} SVG or \href{https://developer.mozilla.org/en/docs/Web/API/CanvasRenderingContext2D}{HTML Canvas}-like path instructions.}\\

{\footnotesize
\begin{minipage}[t]{2.6cm}
    moveTo\\
    quadraticCurveTo\\
    bezierCurveTo
\end{minipage}
\begin{minipage}[t]{1.8cm}
    closePath\\
    arc\\
    rect
\end{minipage}
\begin{minipage}[t]{1.6cm}
    lineTo\\
    arcTo\\
    toString
\end{minipage}
}


%%%%%%%%%%%%%%%%%%%%%%%%%%%%%%%%%%%%%%%%%%%%%%%%%%%
\subsec{\href{https://github.com/d3/d3-shape\#arcs}{Arcs}}{<arc>\textsubscript{G}}
\textit{Generates: circular or annular sectors for use in \href{https://observablehq.com/@d3/pie-chart}{pie} or \href{https://observablehq.com/@d3/donut-chart}{donut} charts, respectively. Here, input data must provide start and end angles.}
\\

{\footnotesize
\begin{minipage}[t]{2.1cm}
    centroid\\
    cornerRadius\\
    padAngle
\end{minipage}
\begin{minipage}[t]{2.1cm}
    innerRadius\\
    startAngle\\
    padRadius
\end{minipage}
\begin{minipage}[t]{1.8cm}
    outerRadius\\
    endAngle\\
    context
\end{minipage}
}


%%%%%%%%%%%%%%%%%%%%%%%%%%%%%%%%%%%%%%%%%%%%%%%%%%%
\subsec{\href{https://github.com/d3/d3-shape\#lines}{Lines}}{\href{https://observablehq.com/@d3/d3-line}{<line>\textsubscript{G}}}
\textit{Generates: a spline (smoothed curve) or polyline (piecewise-connected line) for use in a \href{https://observablehq.com/@d3/line-chart}{line} or \href{https://observablehq.com/@d3/hierarchical-edge-bundling}{edge-bundling} chart. There are two top-level calls: }\texttt{d3.line}\textsuperscript{l}\textit{, and }\texttt{d3.line-radial}\textsuperscript{r}\textit{.}\\

{\footnotesize
\begin{minipage}[t]{2.0cm}
    x\textsubscript{l}\\
    y\textsubscript{l}\\
\end{minipage}
\begin{minipage}[t]{2.0cm}
    defined\\
    curve\\
    context
\end{minipage}
\begin{minipage}[t]{2.0cm}
    angle\textsubscript{r}\\
    radius\textsubscript{r}\\
\end{minipage}
}



%%%%%%%%%%%%%%%%%%%%%%%%%%%%%%%%%%%%%%%%%%%%%%%%%%%
\subsec{\href{https://github.com/d3/d3-shape\#areas}{Areas}}{<area>\textsubscript{G}}
\textit{Generates an area, for use in \href{https://observablehq.com/@d3/area-chart}{area} or \href{https://observablehq.com/@d3/difference-chart}{difference} charts. There are two top-level calls: }\texttt{d3.area}\textsuperscript{a}\textit{ and }\texttt{d3.areaRadial}\textsuperscript{r}\textit{.}\\
{\footnotesize
\begin{minipage}[t]{1.8cm}
    x\textsuperscript{a}\\
    x0\textsuperscript{a}\\
    x1\textsuperscript{a}\\
    y\textsuperscript{a}\\
    y0\textsuperscript{a}\\
    y1\textsuperscript{a}\\
    defined\\
    curve
\end{minipage}
\begin{minipage}[t]{2.0cm}
    context\\
    lineX0\textsuperscript{a}\\
    lineY0\textsuperscript{a}\\
    lineX1\textsuperscript{a}\\
    lineY1\textsuperscript{a}\\
    angle\textsuperscript{r}\\
    startAngle\textsuperscript{r}\\
    endAngle\textsuperscript{r}
\end{minipage}
\begin{minipage}[t]{2.0cm}
    radius\textsuperscript{r}\\
    innerRadius\textsuperscript{r}\\
    outerRadius\textsuperscript{r}\\
    lineStartAngle\textsuperscript{r}\\
    lineInnerRadius\textsuperscript{r}\\
    lineEndAngle\textsuperscript{r}\\
    lineOuterRadius\textsuperscript{r}
\end{minipage}
}



%%%%%%%%%%%%%%%%%%%%%%%%%%%%%%%%%%%%%%%%%%%%%%%%%%%
\subsec{\href{https://github.com/d3/d3-shape\#curves}{Curves}}{d3.curve-\textsubscript{F}}
\textit{These are \underline{not} shapes, but passed to lines \& areas under their }\texttt{<shape>.curve()}\textit{ call. }\texttt{curve}\textit{s are algorithms that, given input data arrays (\say{control points}) yield smooth \href{https://en.wikipedia.org/wiki/Spline\_interpolation}{splines}.}
{\footnotesize
\begin{minipage}[t]{3.0cm}
    Basis\\
    BasisClosed\\
    BasisOpen\\
    Bundle\\
    Cardinal\\
    CardinalClosed\\
    CardinalOpen\\
    CatmullRom\\
    CatmullRomClosed
\end{minipage}
\begin{minipage}[t]{3.0cm}
    CatmullRomOpen\\
    Linear\\
    LinearClosed\\
    MonotoneX\\
    MonotoneY\\
    Natural\\
    Step\\
    StepAfter\\
    StepBefore
\end{minipage}
}
\\


\textit{One can also create \href{https://github.com/d3/d3-shape\#custom-curves}{custom curves}.}


%%%%%%%%%%%%%%%%%%%%%%%%%%%%%%%%%%%%%%%%%%%%%%%%%%%
\subsec{\href{https://github.com/d3/d3-shape\#links}{Links}}{<link>\textsubscript{G}}
\textit{Generate a smooth line segment between passed source and target points for use in \href{https://observablehq.com/@d3/tidy-tree}{tree diagrams}. $\exists$ vertical\textsuperscript{l}, horizontal\textsuperscript{l}, and radial\textsuperscript{r} links.}

{\footnotesize
\begin{minipage}[t]{2.0cm}
    source\\
    target\\
\end{minipage}
\begin{minipage}[t]{2.0cm}
    x\textsuperscript{l}\\
    y\textsuperscript{l}
\end{minipage}
\begin{minipage}[t]{2.0cm}
    context\\
    angle\textsuperscript{r}
    radius\textsuperscript{r}
\end{minipage}
}
\\


%%%%%%%%%%%%%%%%%%%%%%%%%%%%%%%%%%%%%%%%%%%%%%%%%%%
\subsec{\href{https://github.com/d3/d3-shape\#symbols}{Symbols}}{\href{https://observablehq.com/@d3/fitted-symbols}{d3.symbol-\textsubscript{G}}}

\textit{Generate a }\texttt{symbol}\textit{:}\\

\href{https://observablehq.com/@d3/fitted-symbols}{
{\footnotesize
\begin{minipage}[t]{2.0cm}
    Circle\\
    Cross\\
    Wye
\end{minipage}
\begin{minipage}[t]{2.0cm}
    Star\\
    Square\\
\end{minipage}
\begin{minipage}[t]{2.0cm}
    Diamond\\
    Triangle\\
\end{minipage}
}}
\\



\textit{Symbol generators provide the following methods:}\\
{\footnotesize
\begin{minipage}[t]{2.0cm}
    item
\end{minipage}
\begin{minipage}[t]{2.0cm}
    size
\end{minipage}
\begin{minipage}[t]{2.0cm}
    context
\end{minipage}
}
\\


%%%%%%%%%%%%%%%%%%%%%%%%%%%%%%%%%%%%%%%%%%%%%%%%%%%
\subsec{\href{https://github.com/d3/d3-polygon}{Polygons}}{d3.polygon-\textsubscript{G}}

\texttt{d3.polygonHull}\textit{builds a polygon that covers an array of input points. Other top level functions access properties of the resulting polygon:}\\

{\footnotesize
\begin{minipage}[t]{1.5cm}
    Area
\end{minipage}
\begin{minipage}[t]{1.5cm}
    Centroid
\end{minipage}
\begin{minipage}[t]{1.5cm}
    Contains
\end{minipage}
\begin{minipage}[t]{1.5cm}
    Length
\end{minipage}
}

\\
